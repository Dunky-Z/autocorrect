\documentclass[]{article}
% Title Page
\title{Rust程序设计语言}
\centerline{\sc Stupid Stuff I Wish Someone Had Told Me Four Years Ago}
\centerline{\it (Read the .tex file along with this or it won't 
						make much sense)}
\author{Steve Klabnik和Carol Nichols}

\begin{document}
\maketitle

\section{入门指南}
	让我们开始Rust之旅!有很多内容需要学习,但每次旅程总有起点.在本章中,我们会讨论:
	
	1. 在Linux、macOS和Windows上安装Rust
	2. 编写一个打印Hello, world!的程序
	3. 使用Rust的包管理器和构建系统cargo
\section{常见编程概念}
	\subsection{变量和可变性}
	正如第二章中“使用变量储存值” 部分提到的那样,变量默认是不可改变的(immutable)。这是Rust提供给你的众多优势之一,让你得以充分利用Rust提供的安全性和简单并发性来编写代码。不过,你仍然可以使用可变变量。让我们探讨一下Rust为何及如何鼓励你利用不可变性,以及何时你会选择不使用不可变性。
	
	当变量不可变时,一旦值被绑定一个名称上,你就不能改变这个值.为了对此进行说明,使用cargo new variables命令在projects目录生成一个叫做variables的新项目。

	接着,在新建的variables目录,打开src/main.rs并将代码替换为如下代码,这些代码还不能编译,我们会首次检查到不可变错误(immutability error)。

	\subsection{数据类型}
	在Rust中,每一个值都属于某一个数据类型(data type),这告诉Rust它被指定为何种数据,以便明确数据处理方式.
	
具体来说,我们将会学习变量、基本类型、函数、注释和控制流。